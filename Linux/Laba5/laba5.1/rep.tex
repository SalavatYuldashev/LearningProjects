\documentclass{article}
\usepackage[T2A]{fontenc}
\usepackage[utf8]{inputenc}
\usepackage[left=35mm, top=2cm, right=1cm, bottom=20mm, nohead, nofoot]{geometry}
\usepackage[russian]{babel}
\usepackage{indentfirst}
\usepackage{tikz}
\usepackage{graphicx}
\usepackage{listings}
\lstset{
     language=bash,
     basicstyle=\ttfamily
}

\begin{document}

\thispagestyle{empty}	% Отключаем колонтитулы

\begin{center}
	Санкт-Петербургский политехнический университет Петра Великого\\
	Институт компьютерных наук и технологий\\
	\bfseries{«Высшая школа программной инженерии»}
\end{center}

\vspace{20ex} % Задаем размер вертикального промежутка в явном виде

\begin{center}
	\begin{huge} {\bfseries{\scshape курсовая работа}} \end{huge}

	\vspace{3ex}

	{\bfseries Выбор оптимальной опции оптимизации}

	по дисциплине: «Введение в профессиональную деятельность»
\end{center}

\vspace{30ex}

\noindent Выполнил\\
студент гр.в3530904/00021\hfill \begin{minipage}{0.6\textwidth} \hfill С.Н. Юлдашев\end{minipage}

\vspace{3ex}

\noindent Руководитель\\
Ст. преподаватель\hfill \begin{minipage} {0.6\textwidth}\hfill А.В. Петров\end{minipage}

\vspace{3ex}

\hfill \begin{minipage}{0.6\textwidth} \hfill «15» июня 2021 г.\end{minipage}

\vfill

\begin{center}
	Санкт-Петербург\\ 
	2021
\end{center}

\newpage
\tableofcontents
\newpage
\section{Цель работы}
  Определить оптимальную опцию оптимизации.
\section{Задачи}
 \begin{enumerate}
    \item Написать программу на языке C++.
    \item Написать скрипт который выполняет оптимизацию написанной программы.
    \item Выбрать вариант оптимизации дающий наибольшую производительность.
    \item Оформить отчет с использованием LATEX.
  \end{enumerate}
  
\newpage 
\section{Введение}
\setlength{\parindent}{1.25cm}Курсовая работа представляет полное описание выполнения задания по написанию сценария по поиску наиболее подходящей оптимизации.
Целью выполнения сценария — опции оптимизации, оптимальные для заранее созданного приложения.
На этой основе продемонстрированы различные уровни оптимизации, оформленные в виде сценария bash. Bash  - усовершенствованная и модернизированная вариация командной оболочки Bourne shell. Bash в основном соответствует стандарту POSIX, но с рядом расширений.\footnote{Advanced Bash-Scripting Guide - Расширенное руководство по написанию bash-скриптов.}Основной критерий упешной оптимизации - время выполнения программы. \\
\newpage
\section{Основная часть}
Особенности выполнения
\begin{itemize}
\item Приложение без оптимизации обрабатывается 25,645 с.
\item Вычисление занимаемого исполняемым файлом дискового пространства (в байтах) .
\item Сценарий должен принимать только имя исходного файла программы
\end{itemize}
Вывод сценария должен содержать следующую информацию:
\begin{itemize}
\item Текущие опции оптимизации.
\item Время затраченное программой на выполнение.
\item Занимаемое программой дисковое пространство.
\end{itemize}
Программный код сценария start.sh\\[2mm]
\begin{lstlisting}
#!/bin/bash

filename=$1

for i in "-O0" "-Os" "-O1" "-O2" "-O3" \
"-O2 -march=native" "-O3 -march=native" \
"-O2 -march=native -funroll-loops" "-O3 -march=native -funroll-loops" \
"-O3 -march=native -funroll-loops -fipa-cp -flto" \
"-O3 -march=native -funroll-loops -fprofile-generate" \
"-O3 -march=native -funroll-loops -fipa-cp -flto -fprofile-generate"
do
  echo "_____________________"
  echo "  Optimization: $i:"
  echo "....................."
  c++ -Wall -Wextra $i $filename -o prg.veg
  echo "  Time:"
  time ./prg.veg 150 20
  echo "....................."
  echo "  Disk usage:"
  du -b $filename
  echo "_____________________"
done
\end{lstlisting}

\setlength{\parindent}{1.25cm}Для выполнения задания наиболее подходящее решение - это использование цикла, поочередный перебор различных оптимизаций.  Для наглядности алгоритма и удобства восприятия представлена блок - схема, описывающая все шаги выполнения задания.
\begin{center}
\begin{tabular}{ | c | c |  c | }
 \hline Optimization & Time & Size, B \\ 
 \hline -O0 & 0m25,645s & 634 \\  
 \hline -oS & 0m4,985s & 634 \\   
 \hline -O1 & 0m3,705s & 634 \\  
 \hline -O2 & 0m3,745s & 634 \\  
 \hline -O3 & 0m3,611s & 634 \\  
 \hline -O2 -march=native & 0m3,775s & 634 \\  
 \hline -O3 -march=native & 0m3,619s & 634 \\  
 \hline -O2 -march=native -funroll-loops & 0m3,722s
 & 634 \\  
 
 \hline -O3 -march=native -funroll-loops & 0m3,452s & 634 \\  
 \hline -O3 -march=native -funroll-loops -fipa-cp -flto & 0m3,548s & 634 \\  
 \hline -O3 -march=native -funroll-loops -fprofile-generate & 0m4,714s & 634 \\  
 \hline -O3 -march=native -funroll-loops -fipa-cp -flto -fprofile-generate &  0m4,709s & 634 \\  
 \hline
\end{tabular}
\end{center}
\begin{center}Результат программы предствалены в таблице 1\end{center}
%\includegraphics[scale=0.8]{PictureLatex}




\section{ Заключение}
\setlength{\parindent}{1.25cm}Для отимизации времени выполнения программы были использованы 12 методов и комбинаций методов. Наиболее эффективным оказался комбинированный метод оптимизации с оптимальной опцией, межпроцедурной оптимизацией, оптимизацией времени компоновки и с оптимизацией с обратной связью. Разница во времени выполнения программы с худшей и лучшей оптимизации составляет более 3,5 секунд,  очень существенна. Выбор подходящей оптимизации - один из наиболее результативных факторов ее успешности.\newpage
\section{Список Литературы}
\begin{enumerate}
\item Уорд Б. "Внутреннее устройство LINUX"- СПБ.:Питер, 2016. - 384 с.: ил -(Серия "Для профессионалов")
\item Шотс У. "Командная строка LINUX. Полное руководство."- СПБ.:Питер, 2017. - 480 с.: ил -(Серия "Для профессионалов")
\item Cooper M. "Advanced Bash-Scripting Guide" Revision 10 Mar 2014 Revised by: 'PUBLICDOMAIN' release- 910 c
\item URL : https://ru.wikipedia.org/wiki/Bash - статья в Википедии
\end{enumerate}


\newpage 
\section{Приложение 1. Исходный код программы для тестирования.}
\begin{lstlisting}
#include <iostream>
#include <algorithm>
#include <stdlib.h>

    const size_t MB = 1024 * 1024;
    size_t MOD = 0;

    unsigned char uniqueNumber() {
        static unsigned char number = 0;
        return ++number % MOD;
    }

    int main(int argc, char** argv) {
        if (argc < 3) {
            return 1;
        }

        size_t BLOCK_SIZE = atoi(argv[1]) * MB;
        MOD = atoi(argv[2]);

        unsigned char* garbage = (unsigned char*)malloc(BLOCK_SIZE);

        std::generate_n(garbage, BLOCK_SIZE, uniqueNumber);
        std::sort(garbage, garbage + BLOCK_SIZE);

        free(garbage);

        return 0;
    }
\end{lstlisting}

\newpage 
\section{Приложение 2. Блок-схема алгоритма.}
\vspace{20ex}
% Graphic for TeX using PGF
% Title: D:\Data\Linux\Laba5\diag.dia
% Creator: Dia v0.97.2
% CreationDate: Fri Jun 18 16:42:01 2021
% For: Salavat
% \usepackage{tikz}
% The following commands are not supported in PSTricks at present
% We define them conditionally, so when they are implemented,
% this pgf file will use them.
\ifx\du\undefined
  \newlength{\du}
\fi
\setlength{\du}{15\unitlength}
\begin{tikzpicture}
\pgftransformxscale{1.000000}
\pgftransformyscale{-1.000000}
\definecolor{dialinecolor}{rgb}{0.000000, 0.000000, 0.000000}
\pgfsetstrokecolor{dialinecolor}
\definecolor{dialinecolor}{rgb}{1.000000, 1.000000, 1.000000}
\pgfsetfillcolor{dialinecolor}
\pgfsetlinewidth{0.100000\du}
\pgfsetdash{}{0pt}
\pgfsetdash{}{0pt}
\pgfsetbuttcap
\pgfsetmiterjoin
\pgfsetlinewidth{0.100000\du}
\pgfsetbuttcap
\pgfsetmiterjoin
\pgfsetdash{}{0pt}
\definecolor{dialinecolor}{rgb}{1.000000, 1.000000, 1.000000}
\pgfsetfillcolor{dialinecolor}
\pgfpathmoveto{\pgfpoint{18.675009\du}{1.300343\du}}
\pgfpathlineto{\pgfpoint{23.241675\du}{1.300343\du}}
\pgfpathcurveto{\pgfpoint{23.872201\du}{1.300343\du}}{\pgfpoint{24.383342\du}{1.858049\du}}{\pgfpoint{24.383342\du}{2.546015\du}}
\pgfpathcurveto{\pgfpoint{24.383342\du}{3.233981\du}}{\pgfpoint{23.872201\du}{3.791687\du}}{\pgfpoint{23.241675\du}{3.791687\du}}
\pgfpathlineto{\pgfpoint{18.675009\du}{3.791687\du}}
\pgfpathcurveto{\pgfpoint{18.044483\du}{3.791687\du}}{\pgfpoint{17.533342\du}{3.233981\du}}{\pgfpoint{17.533342\du}{2.546015\du}}
\pgfpathcurveto{\pgfpoint{17.533342\du}{1.858049\du}}{\pgfpoint{18.044483\du}{1.300343\du}}{\pgfpoint{18.675009\du}{1.300343\du}}
\pgfusepath{fill}
\definecolor{dialinecolor}{rgb}{0.000000, 0.000000, 0.000000}
\pgfsetstrokecolor{dialinecolor}
\pgfpathmoveto{\pgfpoint{18.675009\du}{1.300343\du}}
\pgfpathlineto{\pgfpoint{23.241675\du}{1.300343\du}}
\pgfpathcurveto{\pgfpoint{23.872201\du}{1.300343\du}}{\pgfpoint{24.383342\du}{1.858049\du}}{\pgfpoint{24.383342\du}{2.546015\du}}
\pgfpathcurveto{\pgfpoint{24.383342\du}{3.233981\du}}{\pgfpoint{23.872201\du}{3.791687\du}}{\pgfpoint{23.241675\du}{3.791687\du}}
\pgfpathlineto{\pgfpoint{18.675009\du}{3.791687\du}}
\pgfpathcurveto{\pgfpoint{18.044483\du}{3.791687\du}}{\pgfpoint{17.533342\du}{3.233981\du}}{\pgfpoint{17.533342\du}{2.546015\du}}
\pgfpathcurveto{\pgfpoint{17.533342\du}{1.858049\du}}{\pgfpoint{18.044483\du}{1.300343\du}}{\pgfpoint{18.675009\du}{1.300343\du}}
\pgfusepath{stroke}
% setfont left to latex
\definecolor{dialinecolor}{rgb}{0.000000, 0.000000, 0.000000}
\pgfsetstrokecolor{dialinecolor}
\node at (20.958342\du,2.746015\du){Begin};
\definecolor{dialinecolor}{rgb}{1.000000, 1.000000, 1.000000}
\pgfsetfillcolor{dialinecolor}
\fill (20.928022\du,13.102255\du)--(25.421532\du,16.074017\du)--(20.928022\du,19.045780\du)--(16.434511\du,16.074017\du)--cycle;
\pgfsetlinewidth{0.100000\du}
\pgfsetdash{}{0pt}
\pgfsetdash{}{0pt}
\pgfsetmiterjoin
\definecolor{dialinecolor}{rgb}{0.000000, 0.000000, 0.000000}
\pgfsetstrokecolor{dialinecolor}
\draw (20.928022\du,13.102255\du)--(25.421532\du,16.074017\du)--(20.928022\du,19.045780\du)--(16.434511\du,16.074017\du)--cycle;
% setfont left to latex
\definecolor{dialinecolor}{rgb}{0.000000, 0.000000, 0.000000}
\pgfsetstrokecolor{dialinecolor}
\node at (20.928022\du,16.269017\du){};
% setfont left to latex
\definecolor{dialinecolor}{rgb}{0.000000, 0.000000, 0.000000}
\pgfsetstrokecolor{dialinecolor}
\node[anchor=west] at (19.267125\du,16.161960\du){if(argc < 3)};
\pgfsetlinewidth{0.100000\du}
\pgfsetdash{}{0pt}
\pgfsetdash{}{0pt}
\pgfsetbuttcap
{
\definecolor{dialinecolor}{rgb}{0.000000, 0.000000, 0.000000}
\pgfsetfillcolor{dialinecolor}
% was here!!!
\pgfsetarrowsend{stealth}
\definecolor{dialinecolor}{rgb}{0.000000, 0.000000, 0.000000}
\pgfsetstrokecolor{dialinecolor}
\draw (20.958342\du,3.791687\du)--(20.928022\du,13.102255\du);
}
\pgfsetlinewidth{0.100000\du}
\pgfsetdash{}{0pt}
\pgfsetdash{}{0pt}
\pgfsetbuttcap
{
\definecolor{dialinecolor}{rgb}{0.000000, 0.000000, 0.000000}
\pgfsetfillcolor{dialinecolor}
% was here!!!
\definecolor{dialinecolor}{rgb}{0.000000, 0.000000, 0.000000}
\pgfsetstrokecolor{dialinecolor}
\draw (16.434511\du,16.074017\du)--(13.702249\du,16.082129\du);
}
\pgfsetlinewidth{0.100000\du}
\pgfsetdash{}{0pt}
\pgfsetdash{}{0pt}
\pgfsetbuttcap
{
\definecolor{dialinecolor}{rgb}{0.000000, 0.000000, 0.000000}
\pgfsetfillcolor{dialinecolor}
% was here!!!
\definecolor{dialinecolor}{rgb}{0.000000, 0.000000, 0.000000}
\pgfsetstrokecolor{dialinecolor}
\draw (25.421532\du,16.074017\du)--(29.614069\du,16.068448\du);
}
% setfont left to latex
\definecolor{dialinecolor}{rgb}{0.000000, 0.000000, 0.000000}
\pgfsetstrokecolor{dialinecolor}
\node[anchor=west] at (14.973094\du,15.687787\du){YES};
% setfont left to latex
\definecolor{dialinecolor}{rgb}{0.000000, 0.000000, 0.000000}
\pgfsetstrokecolor{dialinecolor}
\node[anchor=west] at (26.207697\du,15.748219\du){NO};
\pgfsetlinewidth{0.100000\du}
\pgfsetdash{}{0pt}
\pgfsetdash{}{0pt}
\pgfsetbuttcap
{
\definecolor{dialinecolor}{rgb}{0.000000, 0.000000, 0.000000}
\pgfsetfillcolor{dialinecolor}
% was here!!!
\pgfsetarrowsend{stealth}
\definecolor{dialinecolor}{rgb}{0.000000, 0.000000, 0.000000}
\pgfsetstrokecolor{dialinecolor}
\draw (13.744294\du,16.082129\du)--(13.765316\du,21.653037\du);
}
\pgfsetlinewidth{0.100000\du}
\pgfsetdash{}{0pt}
\pgfsetdash{}{0pt}
\pgfsetbuttcap
\pgfsetmiterjoin
\pgfsetlinewidth{0.100000\du}
\pgfsetbuttcap
\pgfsetmiterjoin
\pgfsetdash{}{0pt}
\definecolor{dialinecolor}{rgb}{1.000000, 1.000000, 1.000000}
\pgfsetfillcolor{dialinecolor}
\pgfpathmoveto{\pgfpoint{19.132458\du}{30.980976\du}}
\pgfpathlineto{\pgfpoint{24.167402\du}{30.980976\du}}
\pgfpathcurveto{\pgfpoint{24.862583\du}{30.980976\du}}{\pgfpoint{25.426138\du}{31.465694\du}}{\pgfpoint{25.426138\du}{32.063624\du}}
\pgfpathcurveto{\pgfpoint{25.426138\du}{32.661554\du}}{\pgfpoint{24.862583\du}{33.146272\du}}{\pgfpoint{24.167402\du}{33.146272\du}}
\pgfpathlineto{\pgfpoint{19.132458\du}{33.146272\du}}
\pgfpathcurveto{\pgfpoint{18.437277\du}{33.146272\du}}{\pgfpoint{17.873722\du}{32.661554\du}}{\pgfpoint{17.873722\du}{32.063624\du}}
\pgfpathcurveto{\pgfpoint{17.873722\du}{31.465694\du}}{\pgfpoint{18.437277\du}{30.980976\du}}{\pgfpoint{19.132458\du}{30.980976\du}}
\pgfusepath{fill}
\definecolor{dialinecolor}{rgb}{0.000000, 0.000000, 0.000000}
\pgfsetstrokecolor{dialinecolor}
\pgfpathmoveto{\pgfpoint{19.132458\du}{30.980976\du}}
\pgfpathlineto{\pgfpoint{24.167402\du}{30.980976\du}}
\pgfpathcurveto{\pgfpoint{24.862583\du}{30.980976\du}}{\pgfpoint{25.426138\du}{31.465694\du}}{\pgfpoint{25.426138\du}{32.063624\du}}
\pgfpathcurveto{\pgfpoint{25.426138\du}{32.661554\du}}{\pgfpoint{24.862583\du}{33.146272\du}}{\pgfpoint{24.167402\du}{33.146272\du}}
\pgfpathlineto{\pgfpoint{19.132458\du}{33.146272\du}}
\pgfpathcurveto{\pgfpoint{18.437277\du}{33.146272\du}}{\pgfpoint{17.873722\du}{32.661554\du}}{\pgfpoint{17.873722\du}{32.063624\du}}
\pgfpathcurveto{\pgfpoint{17.873722\du}{31.465694\du}}{\pgfpoint{18.437277\du}{30.980976\du}}{\pgfpoint{19.132458\du}{30.980976\du}}
\pgfusepath{stroke}
% setfont left to latex
\definecolor{dialinecolor}{rgb}{0.000000, 0.000000, 0.000000}
\pgfsetstrokecolor{dialinecolor}
\node at (21.649930\du,32.263624\du){};
\pgfsetlinewidth{0.100000\du}
\pgfsetdash{}{0pt}
\pgfsetdash{}{0pt}
\pgfsetbuttcap
{
\definecolor{dialinecolor}{rgb}{0.000000, 0.000000, 0.000000}
\pgfsetfillcolor{dialinecolor}
% was here!!!
\definecolor{dialinecolor}{rgb}{0.000000, 0.000000, 0.000000}
\pgfsetstrokecolor{dialinecolor}
\draw (13.774082\du,20.120602\du)--(13.780073\du,32.116656\du);
}
% setfont left to latex
\definecolor{dialinecolor}{rgb}{0.000000, 0.000000, 0.000000}
\pgfsetstrokecolor{dialinecolor}
\node[anchor=west] at (21.145581\du,32.269949\du){end};
\pgfsetlinewidth{0.100000\du}
\pgfsetdash{}{0pt}
\pgfsetdash{}{0pt}
\pgfsetbuttcap
{
\definecolor{dialinecolor}{rgb}{0.000000, 0.000000, 0.000000}
\pgfsetfillcolor{dialinecolor}
% was here!!!
\pgfsetarrowsend{stealth}
\definecolor{dialinecolor}{rgb}{0.000000, 0.000000, 0.000000}
\pgfsetstrokecolor{dialinecolor}
\draw (29.643799\du,16.132637\du)--(29.638937\du,22.744920\du);
}
\pgfsetlinewidth{0.100000\du}
\pgfsetdash{}{0pt}
\pgfsetdash{}{0pt}
\pgfsetbuttcap
{
\definecolor{dialinecolor}{rgb}{0.000000, 0.000000, 0.000000}
\pgfsetfillcolor{dialinecolor}
% was here!!!
\pgfsetarrowsend{stealth}
\definecolor{dialinecolor}{rgb}{0.000000, 0.000000, 0.000000}
\pgfsetstrokecolor{dialinecolor}
\draw (13.745590\du,32.054599\du)--(17.873722\du,32.063624\du);
}
\definecolor{dialinecolor}{rgb}{1.000000, 1.000000, 1.000000}
\pgfsetfillcolor{dialinecolor}
\fill (26.209807\du,22.641111\du)--(26.209807\du,28.230280\du)--(32.868863\du,28.230280\du)--(32.868863\du,22.641111\du)--cycle;
\pgfsetlinewidth{0.100000\du}
\pgfsetdash{}{0pt}
\pgfsetdash{}{0pt}
\pgfsetmiterjoin
\definecolor{dialinecolor}{rgb}{0.000000, 0.000000, 0.000000}
\pgfsetstrokecolor{dialinecolor}
\draw (26.209807\du,22.641111\du)--(26.209807\du,28.230280\du)--(32.868863\du,28.230280\du)--(32.868863\du,22.641111\du)--cycle;
% setfont left to latex
\definecolor{dialinecolor}{rgb}{0.000000, 0.000000, 0.000000}
\pgfsetstrokecolor{dialinecolor}
\node at (29.539335\du,25.630696\du){};
\definecolor{dialinecolor}{rgb}{1.000000, 1.000000, 1.000000}
\pgfsetfillcolor{dialinecolor}
\fill (19.511894\du,6.389911\du)--(23.778870\du,6.389911\du)--(22.280250\du,10.507336\du)--(18.013274\du,10.507336\du)--cycle;
\pgfsetlinewidth{0.100000\du}
\pgfsetdash{}{0pt}
\pgfsetdash{}{0pt}
\pgfsetmiterjoin
\definecolor{dialinecolor}{rgb}{0.000000, 0.000000, 0.000000}
\pgfsetstrokecolor{dialinecolor}
\draw (19.511894\du,6.389911\du)--(23.778870\du,6.389911\du)--(22.280250\du,10.507336\du)--(18.013274\du,10.507336\du)--cycle;
% setfont left to latex
\definecolor{dialinecolor}{rgb}{0.000000, 0.000000, 0.000000}
\pgfsetstrokecolor{dialinecolor}
\node at (20.896072\du,8.643624\du){};
% setfont left to latex
\definecolor{dialinecolor}{rgb}{0.000000, 0.000000, 0.000000}
\pgfsetstrokecolor{dialinecolor}
\node[anchor=west] at (20.224328\du,8.222037\du){Input:};
% setfont left to latex
\definecolor{dialinecolor}{rgb}{0.000000, 0.000000, 0.000000}
\pgfsetstrokecolor{dialinecolor}
\node[anchor=west] at (20.224328\du,9.022037\du){argc};
\definecolor{dialinecolor}{rgb}{1.000000, 1.000000, 1.000000}
\pgfsetfillcolor{dialinecolor}
\fill (11.974073\du,21.613048\du)--(16.500443\du,21.613048\du)--(15.664014\du,23.911119\du)--(11.137644\du,23.911119\du)--cycle;
\pgfsetlinewidth{0.100000\du}
\pgfsetdash{}{0pt}
\pgfsetdash{}{0pt}
\pgfsetmiterjoin
\definecolor{dialinecolor}{rgb}{0.000000, 0.000000, 0.000000}
\pgfsetstrokecolor{dialinecolor}
\draw (11.974073\du,21.613048\du)--(16.500443\du,21.613048\du)--(15.664014\du,23.911119\du)--(11.137644\du,23.911119\du)--cycle;
% setfont left to latex
\definecolor{dialinecolor}{rgb}{0.000000, 0.000000, 0.000000}
\pgfsetstrokecolor{dialinecolor}
\node at (13.819044\du,22.957083\du){};
% setfont left to latex
\definecolor{dialinecolor}{rgb}{0.000000, 0.000000, 0.000000}
\pgfsetstrokecolor{dialinecolor}
\node[anchor=west] at (29.539335\du,25.435696\du){};
% setfont left to latex
\definecolor{dialinecolor}{rgb}{0.000000, 0.000000, 0.000000}
\pgfsetstrokecolor{dialinecolor}
\node[anchor=west] at (13.016344\du,22.672895\du){output:};
% setfont left to latex
\definecolor{dialinecolor}{rgb}{0.000000, 0.000000, 0.000000}
\pgfsetstrokecolor{dialinecolor}
%\node[anchor=west] at (13.016344\du,23.472895\du){"error"};
% setfont left to latex
\definecolor{dialinecolor}{rgb}{0.000000, 0.000000, 0.000000}
\pgfsetstrokecolor{dialinecolor}
\node[anchor=west] at (27.190695\du,24.930292\du){1. create array};
% setfont left to latex
\definecolor{dialinecolor}{rgb}{0.000000, 0.000000, 0.000000}
\pgfsetstrokecolor{dialinecolor}
\node[anchor=west] at (27.190695\du,25.730292\du){2. fill array};
% setfont left to latex
\definecolor{dialinecolor}{rgb}{0.000000, 0.000000, 0.000000}
\pgfsetstrokecolor{dialinecolor}
\node[anchor=west] at (27.190695\du,26.530292\du){3. sort array};
\pgfsetlinewidth{0.100000\du}
\pgfsetdash{}{0pt}
\pgfsetdash{}{0pt}
\pgfsetbuttcap
{
\definecolor{dialinecolor}{rgb}{0.000000, 0.000000, 0.000000}
\pgfsetfillcolor{dialinecolor}
% was here!!!
\definecolor{dialinecolor}{rgb}{0.000000, 0.000000, 0.000000}
\pgfsetstrokecolor{dialinecolor}
\draw (29.512417\du,28.280595\du)--(29.475796\du,32.150985\du);
}
\pgfsetlinewidth{0.100000\du}
\pgfsetdash{}{0pt}
\pgfsetdash{}{0pt}
\pgfsetbuttcap
{
\definecolor{dialinecolor}{rgb}{0.000000, 0.000000, 0.000000}
\pgfsetfillcolor{dialinecolor}
% was here!!!
\pgfsetarrowsend{stealth}
\definecolor{dialinecolor}{rgb}{0.000000, 0.000000, 0.000000}
\pgfsetstrokecolor{dialinecolor}
\draw (29.495073\du,32.093154\du)--(25.426138\du,32.063624\du);
}
\end{tikzpicture}

\end{document}

